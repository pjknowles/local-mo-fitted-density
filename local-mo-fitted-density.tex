\documentclass[journal=jacsat,manuscript=article]{achemso}

\usepackage[version=3]{mhchem} % Formula subscripts using \ce{}

\usepackage{xcolor,soul}
\setlength\marginparwidth{1.5cm}
\usepackage[obeyFinal,textsize=tiny, textwidth=\marginparwidth]{todonotes}
\presetkeys{todonotes}{fancyline, color=blue!30}{}

%%%%%%%%%%%%%%%%%%%%%%%%%%%%%%%%%%%%%%%%%%%%%%%%%%%%%%%%%%%%%%%%%%%%%
%% If issues arise when submitting your manuscript, you may want to
%% un-comment the next line.  This provides information on the
%% version of every file you have used.
%%%%%%%%%%%%%%%%%%%%%%%%%%%%%%%%%%%%%%%%%%%%%%%%%%%%%%%%%%%%%%%%%%%%%
%%\listfiles

%%%%%%%%%%%%%%%%%%%%%%%%%%%%%%%%%%%%%%%%%%%%%%%%%%%%%%%%%%%%%%%%%%%%%
%% Meta-data block
%% ---------------
%% Each author should be given as a separate \author command.
%%
%% Corresponding authors should have an e-mail given after the author
%% name as an \email command. Phone and fax numbers can be given
%% using \phone and \fax, respectively; this information is optional.
%%
%% The affiliation of authors is given after the authors; each
%% \affiliation command applies to all preceding authors not already
%% assigned an affiliation.
%%
%% The affiliation takes an option argument for the short name.  This
%% will typically be something like "University of Somewhere".
%%
%% The \altaffiliation macro should be used for new address, etc.
%% On the other hand, \alsoaffiliation is used on a per author basis
%% when authors are associated with multiple institutions.
%%%%%%%%%%%%%%%%%%%%%%%%%%%%%%%%%%%%%%%%%%%%%%%%%%%%%%%%%%%%%%%%%%%%%
\author{Josh J. Kirsopp}
\author{Athanasios Arvanitidis}
\author{Peter J. Knowles}
\email{KnowlesPJ@Cardiff.ac.uk}
\affiliation[Cardiff University]
{School of Chemistry, Cardiff University, Cardiff CF10 3AT, United Kingdom}


% \title[Orbital-fitted density]
%   {Approximating molecular electrostatic potentials through minimal representations of local molecular orbitals}
\title[Reduced Orbital Potential Approximation]
  {Approximating molecular electrostatic potentials: the Reduced Orbital Potential Approximation}


%\abbreviations{IR,NMR,UV}
%\keywords{American Chemical Society, \LaTeX}


\begin{document}


\begin{tocentry}

Some journals require a graphical entry for the Table of Contents.
This should be laid out ``print ready'' so that the sizing of the
text is correct.

Inside the \texttt{tocentry} environment, the font used is Helvetica
8\,pt, as required by \emph{Journal of the American Chemical
Society}.

The surrounding frame is 9\,cm by 3.5\,cm, which is the maximum
permitted for  \emph{Journal of the American Chemical Society}
graphical table of content entries. The box will not resize if the
content is too big: instead it will overflow the edge of the box.

This box and the associated title will always be printed on a
separate page at the end of the document.

\end{tocentry}


\begin{abstract}
  The abstract.
  Will define
  Distributed Multipole Analysis (DMA).
\end{abstract}


\section{Introduction}
Introduction

  Will define
  Distributed Multipole Analysis (DMA)\cite{Stone1981,Stone1985DistributedAnalysis,Stone2005DistributedSets.}.
Cite some papers\cite{Knizia2015}

\section{Theory}

For an $N$-electron molecule with electronic wavefunction $|\Psi\rangle$ the clamped-nucleus charge density is
\begin{align}
    \rho(\vec r) = \sum_A Z_A \,\delta(\vec r - \vec A)
    -N \,\langle\Psi|\delta(\vec r-\vec r_1)|\Psi\rangle
\end{align}
where $\vec A, Z_A$ denote the positions and charges of the nuclei, and
$\vec r_1$ is the coordinate of one of the electrons.
The electrostatic potential arising from this or any charge density $\rho$ is
\begin{align}
    J[\rho] &= \int d\vec r' \rho(\vec r') |\vec r-\vec r'|^{-1}
\end{align}

% The corresponding electrostatic potential is then
% \begin{align}
    % V(\vec r) & =
    % \sum_A Z_A |\vec r
    % - \vec A|^{-1}+\sum_i J[|\psi_i|^2](\vec r)
% \end{align}
In the distributed multipole approach\cite{Stone1981,Stone1985DistributedAnalysis,Stone2005DistributedSets.} a simple approximation to the potential is
obtained via  a model charge density consisting of point charges, dipole, and higher multipole, moments centred on the atoms.
The quantum-mechanical multipole operators associated with an origin $\vec A$ are\cite{Stone2013}
\begin{align}
    \hat Q_{lm}^A &= \sum_a e_a\, R_{lm}(\vec r_a-\vec A)
    \\
R_{lm}(\vec r)&=\sqrt{\frac{4\pi}{2l+1}}\,r^l \, Y_{lm}(\vec r)
,&
I_{lm}(\vec r)&=\sqrt{\frac{4\pi}{2l+1}}\,r^{-l-1} \, Y_{lm}(\vec r)
\end{align}
where
$e_a, \vec r_a$ are the charges and positions of each particle, and 
$R_{lm}, I_{lm}$ are the regular and irregular solid harmonics\cite{Whittaker1927, Stone2013}.
The potential generated by a unit-magnitude multipole  centred at $\vec A$ is
    $ (-1)^m I_{l,-m}(\vec r-\vec A)$,
% \begin{align}
    % V^A_{lm}(\vec r) = (-1)^m I_{l,-m}(\vec r-\vec A)
% \end{align}
and one assigns
%, by appropriate projection, best-fit and locality criteria
with an appropriate algorithm\cite{Stone1981,Stone1985DistributedAnalysis,Stone2005DistributedSets.}
values $\{Q^A_{lm}\}$ for the site multipole moments, leading to overall model electrostatic potential
\begin{align}
    V^{\text{DMA},L}(\vec r) &= \sum_A \sum_{lm}^L(-1)^m\, Q^A_{lm}\, I_{l,-m}(\vec r-\vec A)
\end{align}
where $L$ is the
maximum angular momentum, $0\le l\le L, -l\le m\le l$, beyond which the multipole expansion is truncated.

In this work, we investigate the possibility of assigning a site of model electronic density and associated potential with each molecular orbital, rather than nuclei (which remain in the model as point nuclear charges only).

For a molecule represented by Hartree-Fock or Kohn-Sham theory, such a model can be exact, since the density is
\begin{align}
    \rho(\vec r) = \sum_A Z_A \delta(\vec r - \vec A)
    -\sum_i |\psi_i(\vec r)|^2
\end{align}
where $\{\psi_i\}$ are the occupied molecular spin-orbitals.
However we furthermore seek a model that, for efficiency reasons, is as computationally
simple as possible, and which is also based on localised orbitals, in order
to ensure stability and uniqueness as the size of the molecule is varied. We assume
from this point forwards that $\{\psi_i\}$ denote orbitals from a Kohn-Sham
or Hartree-Fock calculation rotated using the Pipek-Mezey procedure\cite{Pipek1989a}.

For each orbital, we define an origin that is its charge centroid, i.e., the
choice of origin that gives zero dipole moment for the orbital,
\begin{align}
  \vec r_i = \int d \vec r \,|\psi_i(\vec r)|^2 \,  \vec r
\end{align}
We can then compute the multipole moments arising from the orbital at this origin,
\begin{align}
    Q^i_{lm} = -\int d\vec r \, R_{lm}(\vec r-\vec r_i) \, |\psi_i|^2
    .
\end{align}
Note that by construction there is no dipole ($l=1$) contribution, and $Q^i_{00}=-1$, representing the unit occupancy of spin-orbital $\psi_i$. The first characteristic shape descriptor of the orbital is therefore its quadrupole moment.
Specification of a maximum angular momentum $L$, $0\le l\le L, -l\le m\le l$ is then sufficient to completely define an Orbital Multipole Approximation (OMA) for the potential,
\begin{align}
    V^{\text{OMA},L}(\vec r) &=
    \sum_A Z_A |\vec r
    - \vec A|^{-1}
    % +\sum_i J[|\psi_i|^2](\vec r)
    +\sum_i \sum_{lm}^L(-1)^m\, Q^i_{lm}\, I_{lm}(\vec r - \vec r_i)
    .
\end{align}
Just as with DMA, this model is expected to be faithful, with a convergent series, when evaluated at positions that are sufficiently distant from any of the nuclei or orbital centroids, but omits the
charge-penetration screening of the Coulomb interaction, and for very short $|\vec r-\vec r_i|$ or $|\vec r-\vec A|$ will give divergence. Unlike DMA, OMA is not dependent on any assumption that molecular densities are approximately the sum of near-spherical atomic densities. The localized molecular orbitals are natural centres of charge that arise automatically in the determination of the wavefunction, and there is no need for any artificial construction based
on basis-function centres\cite{Stone1981} or partitioning switch functions\cite{Stone2005DistributedSets.}.
In Table~\ref{tab:1}, we show the exact $V(\vec r)$, together with
$V^{\text{DMA},L}$ and
$V^{\text{OMA},L}$ for several values of $L$, and for several $\vec r$ along the molecular axis of carbon monoxide, corresponding to 1.5, 2.0 and 2.5 van der Waals radii from the nearest atom.
\begin{table}[h]
    \centering
    \begin{tabular}{r|rrrr|}
        $z$ & $V^{\text{exact}}$ 
        &$V^{\text{DMA,4}}$
        &$V^{\text{OMA,2}}$
        &$V^{\text{OMA,4}}$
        \\
        \hline
         $-10 r_{\text{C}}$ % $-14.361919$ 
         & $-0.00061110$
&$-0.0006135$
         \\
         $-5 r_{\text{C}}$ % $-7.180959 $ 
         & $-0.00399768$
&$-0.0040186$
         \\
         $-2.5 r_{\text{C}}$ % $-3.590480$ 
         & $-0.02119799$
&$-0.0268994$
         \\
         $-2 r_{\text{C}}$ % $-2.872384$ 
         & $-0.02433508$
&$-0.0494916$
         \\
         $-1.5 r_{\text{C}}$ % $-2.154288$ 
         & $0.01165262$
&$-0.1076440$
         \\
         $r_e+1.5 r_{\text{O}}$ % $4.002818$ 
         & $0.04446287$
&$-0.0740879$
         \\
         $r_e+2 r_{\text{O}}$ % $4.626428$ 
         & $-0.00623028$
&$-0.0289934$
         \\
         $r_e+2.5 r_{\text{O}}$ % $5.250037$ 
         & $-0.01028582$
&$-0.0154380$
         \\
         $r_e+5 r_{\text{O}}$ % $8.368085$ 
         & $-0.00249035$
&$-0.0025222$
         \\
         $r_e+10 r_{\text{O}}$ % $14.604181$ 
         & $-0.00034611$
&$-0.0003461$
         \\
         \hline
    \end{tabular}
    \caption{Electrostatic potential at points along the molecular axis of CO arising from a B3LYP calculation with the aug-cc-pVQZ basis set\cite{Kendall1992}. C is at the coordinate origin, and O at $(0,0,r_e=1.1282\text\AA)$. }
    \label{tab:1}
\end{table}\todo{truncate decimals in table}
$\dots$.

OMA has more parameters than DMA, but maybe converges more quickly?

Discuss construction of $\{\bar\psi_i\}$.

Damping functions\cite{Koide1981a,Tang1984AnCoefficients,Knowles1986,Knowles1986a,Knowles1987a}

Reduced Orbital Potential Approximation (ROPA) for the potential,
\begin{align}
    V^{\text{ROPA},L}(\vec r) &=
    \sum_A Z_A |\vec r
    - \vec A|^{-1}
    +\sum_i J\left[\left|\bar\psi_i\right|^2\right](\vec r)
    +\sum_i \sum_{lm}^L (-1)^m\,\left(Q^i_{lm}-\bar Q^i_{lm}\right)\, I_{lm}(\vec r - \vec r_i)
    .
\end{align}
\section{Evaluation}

\begin{acknowledgement}

We are grateful to Anthony Stone for helpful discussions.
This work was funded by a Leverhulme Trust Research Project Grant.

\end{acknowledgement}

\begin{suppinfo}


\end{suppinfo}

\bibliography{reaction-orbitals}

\end{document}